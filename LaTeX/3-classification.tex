\chapter{Classification of Comparative Sentences}
\section{The problem}
\label{sec:problem}
\section{Features}
\label{sec:features}
This section presents a summary of features (see table \ref{tbl:features}) which are used to identify comparative arguments.  Each feature falls into one of the following categories (as described in \cite{Aker2017What-works-and-}): \emph{Structural features} capture statistics about tokens and punctuation, as the number of tokens per sentence. \emph{Lexical features} capture statistics on the presence of particular n-grams or verbs. \emph{Syntactic features} represent part-of-speech sequences and their properties. \emph{Indicators show the presence of specific keywords.}
\cite{Aker2017What-works-and-} mentions contextual features as well. Since the data for this thesis consists of isolated sentences, those features are left out.

% Please add the following required packages to your document preamble:
% \usepackage{booktabs}
\begin{table}[h]
\centering
\caption{Classification Features}

\label{tbl:features}
\begin{tabular}{{p{2.5cm}p{5.5cm}p{2cm}p{4cm}}}
\toprule
Name   & Description & Type & Used in \\ \midrule
%STRUCTURAL
number of tokens & Number of tokens in the argumentative component or in the adjacent sentences         & Structural  & \cite{Stab2014Identifying-Arg}     \\ \midrule

punctuation       &  Number of punctuation marks. Boolean feature if the sentences ends with a question mark           &  Structural    &     \cite{Stab2014Identifying-Arg}     \\\midrule
% LEXICAL
n-grams & Boolean features for all uni-, bi- and tri-grams & Lexical & \cite{Stab2014Identifying-Arg}, \cite{Dusmanu2017Argument-Mining}  \\\midrule

WordNet verb synsets & ? & Lexical & \cite{Dusmanu2017Argument-Mining}\\\midrule

verbs and adverbs & Boolean features for words like \enquote{believe} or \enquote{really} & Lexical & \cite{Stab2014Identifying-Arg} \\\midrule

modal verbs & Boolean feature if the sentence contains a modal verb & Lexical & \cite{Stab2014Identifying-Arg} \\\midrule

% STRUCTURAL
structure of the parse tree & depth, number of subclauses & Structural & \cite{Stab2014Identifying-Arg}, \cite{Park2014Identifying-App}  \\ \midrule


% INDICATOR
Discourse markers & Boolean features for the presence of cue words & Indictator & \cite{Stab2014Identifying-Arg}, \cite{Eckle-Kohler2015On-the-Role-of-}, \cite{Park2014Identifying-App} \\ \midrule

% OTHER
Sentiment & Polarity label (positive, negative, neutral) and score & Other &  \cite{Dusmanu2017Argument-Mining} \\
 \bottomrule
\end{tabular}
\end{table}